\documentclass{article}
\usepackage[letterpaper,margin=1in]{geometry}
\usepackage[parfill]{parskip}
\usepackage[utf8]{inputenc}
\usepackage{amsmath}
\usepackage{physics}
\usepackage{bm}
\usepackage[medium]{titlesec}
\DeclareMathOperator{\Lagr}{\mathcal{L}}
\DeclareMathOperator{\Ham}{\mathcal{H}}
\newcommand{\uvec}[1]{\boldsymbol{\hat{\textbf{#1}}}}
\newcommand{\vh}[1]{\hat{\vb{#1}}}

\title{Classical Mechanics}
\author{Matthew Kwiecien}
\begin{document}

\maketitle

\section{Newton's Laws}
\subsection*{Dots and Cross Products}
\begin{equation*}
   \vb{r} \cdot \vb{s} = rs\cos\theta = r_{x}s_{x}+r_{y}s_{y}+r_{z}s_{z}
\end{equation*}
\begin{equation*}
   \vb{r} \times \vb{s} =  (r_{y}s_{z}-r_{z}s_{y})\vh{x} - (r_{x}s_{z}-r_{z}s_{x})\vh{y} + 
   (r_{x}s_{y}-r_{y}s_{x})\vh{z} = 
   \det 
\begin{bmatrix}
    \vh{x} & \vh{y} & \vh{z} \\
    r_{x} & r_{y} & r_{z} \\
    s_{x} & s_{y} & s_{z} \\
\end{bmatrix}
\end{equation*}

\subsection*{Inertial Frames}
An inertial frame is any reference frame in which Newton's first law holds.  Thus, a nonaccelerating, nonrotating frame.

\subsection*{Newton's Second Law}
\begin{center}
\begin{tabular}{cccc}
     Vector Form & 
     Cartesian & 
     2D Polar & 
     Cylindrical Polar 
     \\
     & 
     $(x,y,z)$ & 
     $(r,\phi)$ & 
     $(\rho, \phi, z)$ 
     \\
     \hline
     \\
     $\vb{F} = m \vb{a}$ & 
     \begin{aligned}
     F_{x}  & =m \ddot{x} \\
     F_{y}  & =m \ddot{y} \\
     F_{z}  & =m \ddot{z}
     \end{aligned} & 
     \begin{aligned}
     F_{r}  & =m(\ddot{r}-r\dot\phi^2) \\
     F_{\phi}  & =m(r\ddot{\phi} + 2\dot{r}\dot\phi)
     \end{aligned} & 
     \begin{aligned}
     F_{r}  & =m(\ddot{\rho}-\rho\dot\phi^2) \\
     F_{\phi}  & =m(\rho\ddot{\phi} + 2\dot{\rho}\dot\phi) \\
     F_{z} &= m\ddot{z} \\
     \end{aligned}
     \\
\end{tabular}
\end{center}

\section{Projectiles and Charged Particles}
\subsection*{Linear and Quadratic Drags}
Provided the speed \textit{v} is well below that of sound, the magnitude of the drag force $\vb{f} = -f(v)\vh{v}$ on an object moving through a fluid

$$
f(v) = f_{lin} + f_{quad}
$$
where
$$f_{lin} = bv = \beta Dv$$
$$f_{quad} = cv^2 = \gamma D^2 v^2$$

$D$ is the linear size of the object.  For a sphere, $D$ would be the diameter and $\gamma$, $\beta$ are constants of the fluid.

\subsection*{Lorentz Force on a Charged Particle}
$$
\vb{F} = q(\vb{E} + \vb{v} \times \vb{B})
$$

\section{Momentum and Angular Momentum}
\subsection*{Equation of Motion for a Rocket}
$$
m\dot{v} = -\dot{m}v_{x} + F^{ext}
$$
\subsection*{Center of Mass for Several Particles}
$$
\vb{R} = \frac{1}{M}\sum_{\alpha=1}^{N} m_{\alpha}\vb{r}_{\alpha} = \frac{m_1 \vb{r}_1 + \dots + m_N \vb{r}_N}{M}
$$
where $M = \sum m_{\alpha}$. For a continuous mass distribution

$$
\vb{R} = \frac{1}{M} = \int \vb{r} dm = \frac{1}{M} \int \varrho \vb{r} dV
$$

and $dV$ in spherical 
$$
\int dV f(\vb{r}) = \int r^2 dr \int \sin\theta d\theta \int d\phi f(r,\theta,\phi)
$$
\subsection*{Angular Momentum}
For a particle with position \vb{r} relative to origin $O$ and momentum \vb{p}, the angular momentum about $O$ is
$$ \vb{\ell} = \vb{r} \times \vb{p} $$

For several particles, total angular momentum is
$$
\vb{L} = \sum_{\alpha=1}^{N} \vb{\ell}_{\alpha} = \sum_{\alpha=1}^{N} \vb{r}_{\alpha} \times \vb{p}_{\alpha}
$$
And if all internal forces are central
$$\vb{\dot{L}} = \vb{\Gamma}^{ext}$$
where $\Gamma^{ext}$ is the net external torque, where torque is
$$
\vb{\Gamma} \equiv \vb{\dot{\ell}} = \vb{r} \times \vb{F}
$$

\textbf{Principle of Conservation of Angular Momentum} - If the net external torque on an N-particle system is zero, the system's total angular momentum is constant.

\section{Energy}
\subsection*{Work-KE Theorem}
The change in KE of a particle as it moves from point 1 to 2 is
$$
\Delta T \equiv T_{2} - T_{1} = \int_{1}^{2} \vb{F} \cdot d\vb{r} \equiv W(1 \to 2)
$$
where $T = \frac{1}{2}mv^2$ and $W$ is the work done by the total force $\vb{F}$ on the particle
\subsection*{Conservative Forces and Potential Energy}
A force $\vb{F}$ on a paticle is \textbf{conservative} if 
\begin{enumerate}
    \item it depends only on the particle's position, $\vb{F} = \vb{F(r)}$
    \item for any two points 1 and 2, the work $W(1\to2)$ done by \vb{F} is the same for all paths joining 1 and 2 (or equivalently $\nabla \times \vb{F} = 0$).
\end{enumerate}
If a force $\vb{F}$ is conservative, we can define a corresponding potential energy, so that
$$
U(\vb{r}) = -W(\vb{r}_o \to \vb{r}) \equiv -\int_{\vb{r}_{0}}^{\vb{r}} \vb{F(r')} \cdot d\vb{r'}
$$
and
$$
\vb{F} = -\nabla U
$$
If all forces on a particle are conservative with corresponding potential energies, then the total mechanical energy
$$
E = T + U_1 + \dots + U_n
$$
is constant.  If there are nonconservative forces, $\Delta E = W_{nc}$, the work done by the nonconservative forces.
\subsection*{Central Forces}
A force $\vb{F(r)}$ is \textbf{central} if it is everywhere directed toward or away from a ``force center''. If you take the center to be the origin, then
$$
\vb{F(r)} = f(\vb{r})\uvec{r}
$$
A central force is spherically symmetric 
$$
f(\vb{r}) = f(r)
$$
If and only if it is conservative.

\section{Oscillations}
\subsection*{Hooke's Law}
$$F = -kx \iff U = \frac{1}{2}kx^2$$
\subsection*{Simple Harmonic Motion}
$$
\ddot{x} = -\omega^2 x \iff x(t) = A\cos(\omega t - \delta), \text{etc.}
$$
\subsection*{Damped Oscillations}
If the oscillator is subject to a damping force $-bv$ then,
$$
\ddot{x} + 2\beta \dot{x} + \omega_0^2 x = 0 \iff x(t) = Ae^{-\beta t} \cos(\omega_1 t-\delta)
$$
where $\beta = 2b/m$, $\omega_0 = \sqrt{k/m}$, $\omega_1 = \sqrt{\omega_0^2 - \beta^2}$, and the solution given here is for weak damping ($\beta < \omega_0$)
\subsection*{Driven Damped Oscillations and Resonance}
Given a driving force $F(t) = mf__0\cos(\omega t)$, the motion is 
$$
x(t) = A\cos(\omega t - \delta)
$$
where 
$$
A^2 = \frac{f_0^2}{(\omega_0^2 - \omega^2)^2 + 4\beta^2\omega^2}
$$
This is a transient solution of the homogenous equation, but dies out as time passes.  The long term solution resonates when $\omega$ is close to $\omega_0$.
\subsection*{Fourier Series}
Any periodic driving force can be built as a series of sinusoidal terms, and the resultant motion can be given by 
$$
x(t) = \sum_{n=0}^\infty A_n\cos(n\omega t - \delta_n)
$$
\subsection*{RMS Displacement}
The root mean square displacement 
$$
x_{rms} = \sqrt{\frac{1}{\tau} \int_0^\tau x^2 dt}
$$
is a good measure of the average response of the oscillator and given by Parseval's theorem as
$$
x_{rms} = \sqrt{A_0^2 + \frac{1}{2}\sum_{n=1}^\infty A_n^2}
$$
\section{Calculus of Variations}
\subsection*{Euler-Lagrange Equation}
An integral of the form
$$
S = \int_{x_{1}}^{x_2} f[y(x), y'(x), x]dx
$$
taken along a path $y=y(x)$ is stationary with respect to variations of that path if and only if $y(x)$ satisfies the \textbf{Euler-Lagrange equation}
$$
\frac{\partial f}{\partial y} - \frac{d}{dx}\frac{\partial f}{\partial y'} = 0
$$
For \textit{n} dependent variables, there are \textit{n} equations, e.g.
$$
S = \int_{u_{1}}^{u_2} f[x(u), y(u), x'(u), y'(u), u]du
$$
is stationary with respect to variations of $x(u)$ and $y(u)$ if and only if these two functions satisfy
$$
\frac{\partial f}{\partial x} = \frac{d}{du}\frac{\partial f}{\partial x'}
$$
and
$$
\frac{\partial f}{\partial y} = \frac{d}{du}\frac{\partial f}{\partial y'}
$$
\section{Lagrange's Equations}
\subsection*{The Lagrangian}
The \textbf{Lagrangian} $\Lagr$ of a conservative system 
$$ \Lagr = T - U $$
where \textit{T} and \textit{U} are kinetic and potential energies.

\subsection*{Generalized Coordinates}
The \textit{n} parameters $q_1, \dots, q_n$ are \textbf{generalized coordinates} for an N-particle system if every particle's position $\vb{r}_\alpha$ can be expressed as a function of $q_1, \dots, q_n$ (and time \textit{t}) and vice versa AND $n$ is the smallest number that allows the system to be described in this way.

If $n  < 3N$ in 3 dimensions, the system is said to be \textbf{constrained}.

The coordinates $q_1, \dots, q_n$ are said to be \textbf{natural} if the functional relationships of the $\vb{r}_\alpha$ to $q_1, \dots, q_n$ are independent of time.

The number of \textbf{degrees of freedom} of a system is the number of coordinates that can be independently varied.

If the number of degrees of freedom is equal to the number of generalized coordinates, the system is \textbf{holonomic}.

\subsection*{Lagrange's Equations}
For a holonomic system, Newton's 2nd law is equivalent to the n \textbf{Lagrange Equations}

$$
\frac{\partial \Lagr}{\partial q_i} = \frac{d}{dt}\frac{\partial \Lagr}{\partial \dot q_i} \qquad [i = 1,\dots,n]
$$

and the Lagrange equations are in turn equivalent to Hamilton's principle.

\subsection*{Generalized Momenta and Ignorable Coordinates}
The \textit{i}th \textbf{generalized momentum} $p_i$ is defined to be the derivative
$$
p_i = \frac{\partial \Lagr}{\partial \dot q_i}
$$
If $\partial \Lagr / \partial q_i = 0$, then $q_i$ is \textbf{ignorable} and the generalized momentum is constant.

\subsection*{The Hamiltonian}
The \textbf{Hamiltonian} is defined as 
$$
\Ham = \sum_{i=1}^{n} p_i \dot q_i - \Lagr
$$
If $\partial \Lagr / \partial t = 0$ Then $\Ham$ is conserved; if the coordinates are natural, $\Ham$ is the energy of the system.

\subsection*{Lagrangian for a Charge in an Electromagnetic Field}
For a charge $q$
$$
\Lagr(\vb{r}, \vb{\dot r}, t) = \frac{1}{2} m \vb{\dot r}^2 - q(V - \vb{\dot r} \cdot \vb{A})
$$

\section{Two-Body Central Force Problem}
\subsection*{Relative Coordinate and Reduced Mass}
Re-write in terms of \textbf{relative coordinate}
$$
\vb{r} = \vb{r_1} - \vb{r_2}
$$
and the CM coordinate $\vb{R}$
$$
\vb{R} = \frac{m_1 \vb{r_1} + m_2 \vb{r_2}}{M}
$$
where $M = m_1 + m_2$, thus
$$
\vb{r_1} = \vb{R} + \frac{m_2}{M}\vb{r} \qquad \vb{r_2} = \vb{R} - \frac{m_1}{M}\vb{r}
$$
or if the origin is the CM coordinate then $\vb{R} = 0$, then
$$
\vb{r_1}  = \frac{m_2}{M}\vb{r} \qquad \vb{r_2} = -\frac{m_1}{M}\vb{r}
$$
and the 2-body problem is reduced to a problem of two independent particles, a free particle with mass $M = m_1 + m_2$ and position $\vb{R}$, and a particle with mass equal to the \textbf{reduced mass}
$$
\mu = \frac{m_1  m_2}{m_1 + m_2}
$$
position $\vb{r}$ and potential energy $U(r)$.

Then, the Lagrangian can be written as
$$
\Lagr = \Lagr_{CM} + \Lagr_{Rel} = \frac{1}{2}M\dot{\vb{R}}^2 + \left( \frac{1}{2}\mu \dot{\vb{r}}^2 - U(\vb{r})\right)
$$

To obtain $U(\vb{r})$ you must use the $\vb{r_1}$, $\vb{r_2}$ relations above (these were used to derive the kinetic energy formulation).


\subsection*{The Equivalent One-Dimensional Problem}
The motion of the relative coordinate, with a given angular momentum $\ell$, is eqivalent to the motion of a particle in one radial dimension, with mass $\mu$, position $r$ (with $0 < r < \infty)$, and \textbf{effective potential energy}
$$
U_{eff}(r) = U(r) + U_{cf}(r) = U(r) + \frac{\ell^2}{2\mu r^2}
$$
where $U_{cf}$ is called the \textbf{centrifugal potential energy}.
\subsection*{The Transformed Radial Equation}
Let $u = 1/r$ and replace $t$ with $\phi$, then we get the equation of the one-dimensional radial motion:
$$
u''(\phi) = -u(\phi) - \frac{\mu}{\ell^2 u(\phi)^2} F
$$
\subsection*{The Kepler Orbits}
For $F = Gm_1 m_2/r^2 = \gamma/r^2$, the solution to the radial equation is
$$
r(\phi) = \frac{c}{1+\epsilon\cos\phi}
$$
from here you can see the min/max radius, and where $c = \ell^2/\gamma\mu$ and $\epsilon$ is related to the energy by 
$$
E = \frac{\gamma^2 \mu}{2 \ell^2}(\epsilon^2 - 1)
$$

Equation of an ellipse:
$$
\frac{(x-c)^2}{a^2} + \frac{y^2}{b^2} = 1
$$

\textbf{Perigee} is $r_{min}$ \textbf{Apogee} is $r_{max}$, and since 
$$
\frac{r_{min}}{r_{max}} = \frac{1-\epsilon}{1+\epsilon}
$$

Period of orbit $\tau$:
$$
\tau^2 = \frac{4\pi^2}{GM} a^3
$$

This \textbf{Kepler Orbit} is an ellipse, parabola, or hyperbola, according as the eccentricity $\epsilon$ is less than, equal to, or greater than 1.

\section{Mechanics in Noninertial Frames}
\subsection*{Inertial Force in an Accelerating but Nonrotating Frame}
The motion of a body, as seen in a frame that has acceleration $\vb{A}$ relative to an inertial frame, can be found using Newton's second law in the form $ m \ddot{\vb{r}} = \vb{F} + \vb{F}_{inertial}$, where $\vb{F}$ is the net force on the body (as measured in any inertial frame) and $\vb{F}_{inertial}$ is an additional \textbf{inertial force}
$$
\vb{F}_{inertial} = -m\vb{A}
$$
\subsection*{The Angular Velocity Vector}
For a body rotating about an axis specified by the unit vector $\vb{u}$ at a rate $\omega$, its \textbf{angular velocity vector} is
$$
\bm{\omega} = \omega \vb{u}
$$
\subsection*{``Useful Relation''}
The velocity of a point $\vb{r}$ fixed in a rigid body that is rotating with angular velocity $\bm{\omega}$ is 
$$
\vb{v} = \bm{\omega} \times \vb{r}
$$

\subsection*{Time Derivatives in a Rotating Frame}
If a frame $\mathcal{S}$ has angular velocity $\bm{\Omega}$ relative to frame $\mathcal{S}_0$, then the time derivatives of a single vector $\vb{Q}$ as seen in the two frames are related by

$$
\left( \frac{d\vb{Q}}{dt} \right)_{\mathcal{S}_0} = \left( \frac{d\vb{Q}}{dt} \right)_{\mathcal{S}} + \vb{\Omega} \times \vb{Q}
$$

\subsection*{Newton's Second Law in a Rotating Frame}
If a frame $\mathcal{S}$ has angular velocity $\vb{\Omega}$ relative to an inertial frame $S_{0}$, then Newton's second law in the rotating frame takes the form
$$
m\ddot{\vb{r}} = \vb{F} + \vb{F}_{cor} + \vb{F}_{cf}
$$
where \textbf{F} is the net force on the body as measured in any inertial frame, and the inertial forces $\vb{F}_{cor}$ and $\vb{F}_{cf}$ are the \textbf{Coriolis} and \textbf{centrifugal forces}

$$
\vb{F}_{cor} = 2m\dot{\vb{r}} \times \vb{\Omega} \qquad \text{and} \qquad \vb{F}_{cf} = m(\vb{\Omega} \times \vb{r}) \times \vb{\Omega}
$$

\subsection*{Free-Fall Acceleration}
The observed free-fall acceleration $\vb{g}$ (defined as the initial acceleration, relative to the earth from rest) includes the ``true" gravitational acceleration $\vb{g_0}$ and the effect of the centrifugal force
$$
\vb{g} = \vb{g_0} + (\vb{\Omega} \times \vb{R}) \times \vb{\Omega}
$$
``Vertical" is defined as the direction of $\vb{g}$ and ``horizontal'' as perpendicular to $\vb{g}$
\section{Rotational Motion of Rigid Bodies}
\subsection*{CM and Relative Motions}
$$
\vb{L} = \vb{L}\text{(motion of CM)} + \vb{L}\text{(motion relative to CM)}
$$
and 
$$
T = T\text{(motion of CM)} + T\text{(motion relative to CM)}
$$
\subsection*{The Moment of Inertia Tensor}
The angular momentum $\vb{L}$ and the angular velocity $\bm\omega$ of a rigid body are related by 
$$
\vb{I} = 
\begin{bmatrix}
    I_{xx} & I_{xy} & I_{xz}\\
    I_{yx} & I_{yy} & I_{yz}\\
    I_{zx} & I_{zy} & I_{zz}\\
\end{bmatrix}
$$

$$
\vb{L} = \vb{I}\bm\omega
$$
where $\vb{L}$ and $\bm\omega$ must be seen as $3 \times 1$ columns and $\vb{I}$ is the $3\times3$ \textbf{moment of inertia tensor}, whose diagonal and off-diagonal elements are defined as 
$$
I_{xx} = \sum_{\alpha} m_{\alpha}(y_{\alpha}^2 + z_{\alpha}^2) \qquad \text{and} \qquad I_{xy} = -\sum_{\alpha} m_{\alpha}x_{\alpha}y_{\alpha}
$$
etc.

In integral form
$$
I_{xx} = \int_V dV \varrho \rho^2
$$
More compactly, we have

$$
I_{ij} = \int \varrho (r^2 \delta_{ij} - r_i r_j) dV
$$

Now, if you get the moment of inertia tensor relative to the center of mass, you can find the moment of inertia tensor about a point P displaced from the center of mass by $(a,b,c)$ using the relation

$$
I_{xx} = I_{xx}^{cm} + M(b^2 + c^2) \qquad I_{xy} = I_{xy}^{cm} - M(ab)
$$

Where $\rho^2$ is the distance from the axis of rotation, and $dV$ in spherical 
$$
\int dV = \int r^2 dr \int \sin\theta d\theta \int d\phi
$$


\subsection*{Principal Axes}
A \textbf{principal axis} of a body (about a point $O$) is any axis through $O$ with the property that if $\bm\omega$ points along the axis, then $\vb{L}$ is parallel to $\bm\omega$, e.g.
$$
\vb{L} = \lambda \bm\omega
$$
for some real number $\lambda$.  For any body and any point $O$, there are three perpendicular principal axes through $O$.

With respect to it's principal axes, the inertial tensor has the \textbf{diagonal form}

$$
\vb{I'} = \begin{bmatrix}
    \lambda_1 & 0 & 0 \\
    0 & \lambda_2 & 0 \\
    0 & 0 & \lambda_3 \\
  \end{bmatrix}
$$
\subsection*{Euler's Equations}
If $\vb{\dot{L}}$ denotes the rate of change of a body's angular momentum as seen in a frame fixed in the body (body frame), then it satisfies \textbf{Euler's equations}
$$
\vb{\dot{L}} + \bm \omega \times \vb{L} = \bm \Gamma
$$
\subsection*{Euler's Angles}
The orientation of a rigid body can be specified by the three \textbf{Euler Angles} $\theta, \phi, \psi$ 

The Lagrangian for a rigid body spinning about a fixed pivot is
$$
\Lagr = \frac{1}{2} \lambda_1 (\dot\phi^2 \sin^2\theta + \dot\theta^2) + \frac{1}{2} \lambda_3(\dot\psi + \dot\phi \cos\theta)^2 - M g R \cos\theta
$$

\section{Coupled Oscillators and Normal Modes}

\section{Hamiltonian Mechanics}
Let $q$ be a generalized coordinate and $p = \partial \Lagr / \partial q$ be the generalized momentum.  Then the Hamiltonian is

$$\Ham = \Ham(q,p,t) = \sum_{i=1}^{n} p_i \dot{q_i} - \Lagr $$

and

$$
\dot{q_i} = \frac{\partial \Ham}{\partial p_i} \qquad \dot{p_i} = - \frac{\partial \Ham}{\partial q_i} 
$$

If the generalized coordinates are independent of time $t$, then the Hamiltonian $\Ham$ is just the total energy $T + V$

\section{Collision and Scattering}

\section{Special Relativity}

\textbf{Inertial Frame}: Any reference frame (e.g. system of coordinates and time) in which all the laws of physics hold in their usual form.

\textbf{First Postulate of Relativity:} If $S$ is an inertial frame and if a second frame $S^'$ moves with constant velocity relative to $S$, then $S'$ is also an inertial frame.

\textbf{Second Postulate of Relativity:} The speed of light in vacuum has the same value $c$ in every direction in all inertial frames.

Everything basically comes from the above.
\subsection{Time Dilation}
For two events observed in frame $\mathcal{S}_0$, separated by time $\Delta t_0$, the time between them in frame $\mathcal{S}$ is
$$
\Delta t = \gamma \Delta t_0
$$
where
$$
\gamma = \frac{1}{\sqrt{1-\beta^2}}, \qquad \beta = \frac{V}{c}
$$
and $V$ is the speed of $\mathcal{S}$ relative to $\mathcal{S}_0$.

\subsection{Length Contraction}
For a body with rest length $l_0$, a frame $\mathcal{S}$ traveling with velocity $V$ in the direction of the length is 
$$
l = \frac{l_0}{\gamma}
$$
Note that lengths perpendicular to the velocity are unchanged.

\subsection{Lorentz Transformations}

Events measured in two frames are related by 

$$
x' = \Lambda x
$$
where 
$$
\Lambda = \begin{bmatrix}
    \gamma & 0 & 0 & -\gamma \beta \\
    0 & 1 & 0 & 0 \\
    0 & 0 & 1 & 0 \\
    -\gamma \beta & 0 & 0 & \gamma \\
  \end{bmatrix}
$$
which gives 
\begin{align*}
    x'_1 = & \gamma x_1 - \gamma \beta x_4 \\
    x'_2 = & x_2 \\
    x'_3 = & x_3 \\
    x'_4 = & -\gamma \beta x_1 + \gamma x_4 
\end{align*}
for a four vector $$x = (x_1, x_2, x_3, x_4) = (\vb{x}, ct)$$ To obtain the inverse transform, exchange prime and unprimed variables and change the sign of $B$.  

\subsection{Scalar Product}
The scalar product of a four-vector is given by 
$$
x \cdot y = x_1 y_1 + x_2 y_2  + x_3 y_3  - x_4 y_4 
$$
and is invariant under all Lorentz transformations.

\subsection{Doppler Effect}
Light from a source traveling with velocity $V$ relative to frame $S$ is observed at an angle $\theta$, the angle between the velocity and the ray of light. For a rest frame frequency $\omega_0$, the frequency observed in frame $\mathcal{s}$ is
$$
\omega = \frac{\omega_0}{\gamma ( 1-\beta \cos \theta)}
$$

\subsection{Mass, Four-Velocity, Momentum, Energy}
\begin{itemize}
    \item The \textbf{invariant mass} is defined as it's \textbf{rest mass}
    \item The \textbf{four-velocity} is $u = \frac{dx}{dt_0} = \gamma ( \vb{v}, c)$
    \item the \textbf{four-momentum} is
    $p = mu = (\gamma m \vb{v}, \gamma m c) = (\vb{p}, E/c)$
    \item The \textbf{three-force} and \textbf{four-force} are given by $\vb{F} = d\vb{p}/dt$ and $K = dp/dt_0$
\end{itemize}

Re-arranging the above, we have 

$$
\beta = \frac{\vb{p}c}{E}, \quad p \cdot p = -(mc)^2, \quad E^2 = (mc^2)^2 + (\vb{p}c)^2
$$

\subsection{Massless Particles}
Obviously, when $m = 0$, from the above we have
$E = |\vb{p}c|, \quad v = c, \quad p^2 = 0$

\end{document}
