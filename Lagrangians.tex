\documentclass{article}
\usepackage[parfill]{parskip}
\usepackage[utf8]{inputenc}
\usepackage{amsmath}
\DeclareMathOperator{\Lagr}{\mathcal{L}}
\newcommand{\uvec}[1]{\boldsymbol{\hat{\textbf{#1}}}}
\title{Taylor 7.29}
\author{Matthew Kwiecien}
\begin{document}

\maketitle

\section{Problem}
Figure 7.14 shows a simple pendulum (mass \textit{m}, length \textit{l}) whose point of support \textit{P} is attached to the edge of a wheel (center \textit{O}, radius \textit{R}) that is forced to rotate at a fixed angular velocity $\omega$.  At \textit{t = 0}, the point \textit{P} is level with \textit{O} on the right.  Write down the Lagrangian and find the equation of motion for the angle $\phi$.

\section{Solution}

The first step is to obtain the position of the bob of the pendulum as a function of time.  Once we obtain this position as a function of time, we can differentiate it with respect to time to obtain the velocity, which we can use for the kinetic energy of the system.

First, take the triangle formed by the pendulum with the string length \textit{l} as the hypotenuse.  From geometry we have the $x,y$ position of the bob as 

$$
x = l\sin\phi \textrm{, } y= -l\cos\phi
$$

where $y$ is negative because we have taken the point $O$ to be the zero for potential energy.

Now, we need to include the $x,y$ contributions from the rotating disk.  Observe that the angle between the zero state of the system and the current position of the point $P$ is $\omega t$.  Given this, from geometry we have

$$
x = R\cos\omega t \textrm{, } y = R\sin\omega t
$$

Combining this with the above, we get the position of the pendulum bob in terms of $\phi$ and $t$ as:

$$
r(t) = (l\sin\phi + R\cos\omega t) \uvec{x} - (l\cos\phi + R\sin\omega t) \uvec{y}
$$

Now, we can obtain the potential energy of the system by finding $T=\frac{1}{2}m\dot{\vec{r}}^2$. Differentiating $r(t)$ and squaring, we get

$$
T = \frac{1}{2}m(\dot{\phi}^2l^2+\omega^2R^2+2\dot{\phi}l\omega R(\sin\phi\cos\omega t- \cos\phi\sin\omega t)
$$

To simplify we use the identity 
$$
\sin x\cos y - \cos x \sin y = \sin(x - y)
$$
Thus
$$
T = \frac{1}{2}m(\dot{\phi}^2l^2+\omega^2R^2+2\dot{\phi}l\omega R\sin(\phi - \omega t))
$$
Now for the potential energy $u = mgy$, we already found $y$ in terms of $\phi$, substituting:
$$
U = mgy = mg(-l\cos\phi + R\sin\omega t)
$$

Now, combining $T$ and $U$ we obtain the Lagrangian, $\Lagr$:
$$
\Lagr = \frac{1}{2}m(\dot{\phi}^2l^2+\omega^2R^2+2\dot{\phi}l\omega R\sin(\phi - \omega t)) + mgl\cos\phi - mgR\sin\omega t
$$

To obtain the equations of motion, we find the function which satisfies the Euler-Lagrange equation
$$
\frac{\partial \Lagr}{\partial \phi} = \frac{d}{dt}\frac{\partial \Lagr}{\partial \dot{\phi}}
$$

For the left hand side we have

$$
\frac{\partial \Lagr}{\partial \phi} = -mgl\sin\phi + \dot{\phi}ml\omega R\cos(\phi-\omega t)
$$

and the right hand side:

$$
\frac{d}{dt}\frac{\partial \Lagr}{\partial \dot{\phi}} = \frac{d}{dt} ml^2 \dot{\phi} + ml\omega R\sin(\phi - \omega t)
$$

Differentiating, we get

$$
\frac{d}{dt}\frac{\partial \Lagr}{\partial \dot{\phi}} = ml^2 \ddot{\phi} + \dot{\phi}ml\omega R\cos(\phi - \omega t) -  \omega^2 mlR\cos(\phi - \omega t)
$$

Setting the two equal, we see

$$
-mgl\sin\phi + \dot{\phi}ml\omega R\cos(\phi-\omega t) = ml^2 \ddot{\phi} + \dot{\phi}ml\omega R\cos(\phi - \omega t) -  \omega^2 mlR\cos(\phi - \omega t)
$$

Simplifying and solving for $\ddot{\phi}$ we get

$$
\ddot{\phi} = -\frac{g}{l}\sin\phi + \frac{\omega^2R}{l}\cos(\phi - \omega t)
$$

Notice that in the special case of $\oemga = 0$, our equation simplifies to 

$$
\ddot{\phi} = -\frac{g}{l}\sin\phi
$$

which is of course, a simple pendulum.



\end{document}
