\documentclass{article}
\usepackage[letterpaper,margin=1in]{geometry}
\usepackage[parfill]{parskip}
\usepackage[utf8]{inputenc}
\usepackage{amsmath}
\usepackage{physics}
\usepackage{braket}
\usepackage{siunitx}
\usepackage{bm}
\DeclareMathOperator{\Lagr}{\mathcal{L}}
\DeclareMathOperator{\Ham}{\mathcal{H}}
\usepackage{calligra}

\DeclareMathAlphabet{\mathcalligra}{T1}{calligra}{m}{n}
\DeclareFontShape{T1}{calligra}{m}{n}{<->s*[2.2]callig15}{}
\newcommand{\scriptr}{\mathcalligra{r}\,}
\newcommand{\boldscriptr}{\pmb{\mathcalligra{r}}\,}
\newcommand{\uvec}[1]{\boldsymbol{\hat{\textbf{#1}}}}
\newcommand{\vh}[1]{\hat{\vb{#1}}}
\newcommand{\ck}{\frac{1}{4\pi\epsilon_0}}

\title{Electricity and Magnetism Review}
\author{Matthew Kwiecien}
\begin{document}

\maketitle

\section{Potentials}

\textbf{Curl-less fields} (or irrotational). 
$$
\nabla \times \vb{F} = 0 \iff \vb{F} = -\nabla V
$$
All are equivalent:
\begin{itemize}
    \item $\nabla \times \vb{F} = \vb{0}$
    \item $\int_a^b F \cdot d\vb{l}$ is independent of path for any given endpoints
    \item $\oint \vb{F} \cdot d\vb{l} = 0$ for any closed loop
    \item $\vb{F}$ is the gradient of some scalar function $\vb{F} = -\nabla V$
\end{itemize}

\textbf{Divergence-less fields} (or solenoidal). 
$$
\nabla \cdot \vb{F} = 0 \iff \vb{F} = \nabla \times \vb{A}
$$
for a vector potential $\vb{A}$.  All are equivalent:
\begin{itemize}
    \item $\nabla \cdot \vb{F} = 0$ everywhere
    \item $\int \vb{F} \cdot d\vb{a}$ is independent of surface for any given boundary line
    \item $\oint \vb{F} \cdot d\vb{a} = 0$ for any closed surface
    \item $\vb{F} $ is the curl of some vector function $\vb{F} = \nabla \times \vb{A}$
\end{itemize}

Regardless, any vector field can be written as
$$
\vb{F} = -\nabla V + \nabla \times \vb{A}
$$

\section{Electrostatics}

Start with \textbf{Coulomb's Law} which gives you the force on a test charge $Q$ due to a single point charge $q$ at rest a distance $\scriptr$ away
$$
\vb{F} = \ck \frac{qQ}{\scriptr^2} \vh{\boldscriptr}
$$
\subsection{Calculating the E Field}
For a single point charge at the origin
$$
\vb{E}(\vb{r}) = \ck \frac{q}{r^2} \vh{r}
$$
And for a continuous charge distribution
$$
\vb{E}(\vb{r}) = \ck \int \frac{\rho(\vb{r})}{r^2} \vh{r} d\tau
$$

In sum, we have the following tools for translating equations from point, line, surface, and volume
$$
\sum_{i=1}^n (\,)q_i \sim \int_{line} (\,)\lambda dl \sim \int_{surface} (\,) \sigma da \sim \int_{volume} (\,) \rho d\tau
$$

\subsection{Gauss's Law}
Always use Gauss's law if possible to find the E field exploiting symmetries:
$$
\oint \vb{E} \cdot d\vb{a} = \frac{1}{\epsilon_0} Q_{enc} = \int_\mathcal{V} \frac{\rho}{\epsilon_0} d\tau
$$

There are only 3 symmetries to exploit
\begin{itemize}
    \item \textbf{Spherical} - use a concentric sphere
    \item \textbf{Cylindrical} - use a coaxial cylinder
    \item \textbf{Plane} - Use a pillbox that straddles the surface
\end{itemize}

Then you just crank the lever:
\begin{enumerate}
    \item Calculate $Q_{enc}$
    \item Use symmetry to pull $\abs{E}$ out of the integral 
    \item Always remember that the Gaussian surface is not defined in terms of constant variables, they vary, e.g. use $r$ not $R$ as the radius.
\end{enumerate}

\subsection{Calculating the Potential}
Once you have the E field, it's easy 

$$
V(\vb{r}) = - \int_{\infty}^{r} \vb{E} \cdot d\vb{l}, \qquad \vb{E} = - \nabla V
$$

The lower bound is your reference point, which is infinity (so $V(\infty) = 0$), but \textbf{only if you do not have an infinitely long/large charge distribution}.  If you do have an infinity long wire, say, you need to set $V(a) = 0$ instead of infinity.

Using differential form of Gauss's law, you can get \textbf{Poisson's equation}

$$
\nabla^2 V = -\frac{\rho}{\epsilon_0}
$$

and if there's no charge it reduces to Laplace's equation

$$
\nabla^2 V = 0
$$

The electrostatic \textbf{boundary conditions} are

\begin{enumerate}
    \item $\vb{E}^{||}_{above} = \vb{E}^{||}_{below}$
    \item $\vb{E}^{\perp}_{above} - \vb{E}^{\perp}_{below} = \frac{\sigma}{\epsilon_0}\vh{n}$
    \item $V_{above} = V_{below}$
    \item $\frac{\partial V_{above}}{\partial n} - \frac{\partial V_{below}}{\partial n}  = -\frac{1}{\epsilon_0}\sigma $
\end{enumerate}

\subsection{Work (or Energy)}

For point charges
$$
W = \frac{1}{2} \sum_{i=1}^n q_iV(\vb{r}_i)
$$
And for a charge density
$$
W = \frac{1}{2} \int \rho V d\tau
$$
and utilizing Gauss' law we can derive
$$
W = \frac{\epsilon_0}{2} \int E^2 d\tau
$$

\subsection{Conductors and Capacitors}
Important facts about conductors
\begin{itemize}
    \item E = 0 inside a conductor
    \item $\rho = 0$ inside a conductor
    \item Any net charge is on the surface
    \item A conductor is an equipotential
    \item E is perpendicular to the surface outside the conductor
\end{itemize}

Capacitance is purely geometric and describes the proportionality of Q to V for two conductors.

In sum, you can get the electric field using Coulomb's law, or using the equation for a potential, or using Poisson's equation.

\section{Potentials}

\subsection{Laplace's Equation}
Electrostatics is basically the study of Laplace's Equation
$$
\nabla^2 V = 0
$$

\begin{itemize}
    \item It basically is the straightest line between two points, no local minima or maxima.
    \item To solve Laplace's equation, start with boundary conditions (as given by  problem), and use separation of variables.
    \item In cartesian coordinates, the solutions are fourier series.
    \item In spherical coordinates, the solutions are the Legendre Polynomials.
\end{itemize}

\subsection{Method of Images}
The uniqueness theorems of Poisson's equation guarantees that if you can find any potential that satisfies the boundary conditions, THAT is the only potential which can describe the system.

\begin{itemize}
\item The image has to have opposite charge
\item You cannot put image charges in the region where you are calculating $V$
\end{itemize}

\subsection{Multipole expansion}

This expresses an electric potential from very far away in terms of $\frac{1}{r}$

\begin{align*}
V(\vb{r}) & = \ck \sum_{n=0}^{\infty} \frac{1}{r^{(n+1)}} \int (r')^n P_n (\cos\alpha)\rho(\vb{r}') d\tau'\\
& = \ck \left[ \frac{1}{r} \int \rho(\vb{r}')d\tau' + \frac{1}{r^2} \int r' \cos \alpha \rho(\vb{r}') d\tau' + \frac{1}{r^3}\int(r')^2 \left(\frac{3}{2} \cos^2 \alpha - \frac{1}{2} \right) \rho(\vb{r}') d\tau' + \dots \right]
\end{align*}

Where the terms above go monopole, dipole, quadrupole, etc.

\begin{itemize}
    \item Usually the monopole term dominates very far from the origin.
    \item If the total charge is zero, the dipole term will dominate.
\end{itemize}

The dipole moment is given by 
$$
\vb{p} = \int \vb{r}' \rho(\vb{r}') d\tau'
$$
and the potential becomes
$$
V_{dip}(\vb{r}) = \ck \frac{\vb{p} \cdot \vh{r}}{r^2}
$$
and
$$
\vb{p} = \sum_{i=1}^n q_i \vb{r}'_i = q\vb{d}
$$

\textbf{Physical vs perfect dipole}: A perfect dipole has potential given exactly by $V_{dip}$ and requires $\vb{d} \then 0$ but requires infinite $q$.  A physical dipole is one you can actually construct.

\section{Electric Fields in Matter}
This is all about insulators or \textbf{dielectrics}.  The main difference here is that while conductors can completely cancel out the E field, dielectrics only can \textit{kinda}.

\subsection{Polarization}
For a dielectric, you get a bunch of little dipoles pointing in the direction of the field, thus you get the \textbf{polarization}
$$
P = \textit{dipole moment per unit volume}
$$

\textbf{Bound Charges} are charges that are essentially accumulated on the surface (or in the volume if it's not constant inside).  Think of dielectrics as two overlaid spheres of charges (+, -), and bound charges are the outer left over charges when those spheres slightly shift in an E Field.

For this setup use Gauss's law and, you can see that the charge density of bound charges is
$$
\rho_b = -\nabla \cdot \vb{P}
$$

\subsection{The Electric Displacement}
We have two things that contribute to the electric field now \textbf{bound charges} and \textbf{everything else}.  The charge density can be written as

$$
\rho = \rho_b + \rho_f
$$
Combining with Gauss's Law, we get 
$$
\epsilon_0 \nabla \cdot \vb{E} = \rho = \rho_b + \rho_f = - \nabla \cdot \vb{P} + \rho_f
$$
Re-arranging 
$$
\nabla \cdot (\epsilon_0 \vb{E} + \vb{P}) = \rho_f
$$
and let the \textbf{electric displacement} be defined as $\vb{D} = \epsilon_0 \vb{E} + \vb{P}$

Then,
$$
\nabla \cdot \vb{D} = \rho_f \qquad \oint \vb{D} \cdot d\vb{a} = Q_f_{enc}
$$

The electrostatic \textbf{boundary conditions} in terms of $\vb{D}$ are

\begin{enumerate}
    \item $D^{\perp}_{above} - D^{\perp}_{below} = \sigma_f$
    \item $\vb{D}^{||}_{above} - \vb{D}^{||}_{below} = \vb{P}^{||}_{above} - \vb{P}^{||}_{below}$
\end{enumerate}

That is basically all of the effects of polarization.

\subsection{Linear Dielectrics}
Materials that follow 
$$
\vb{P} = \epsilon_0 \chi_e \vb{E}
$$
are linear dielectrics.

\begin{itemize}
    \item $\chi_e$ - \textbf{Electric Susceptibility}: Depends on microscopic structure of substance
    \item $\epsilon = \epsilon_0(1+\chi_e)$
    \item $\epsilon$ - \textbf{permittivity} of a material
    \item If $\chi_e = 0$, you're in ``free space'' and that's how get you $\epsilon_0$
    \item $\epsilon_r = 1 + \chi_e = \frac{\epsilon}{\epsilon_0}$
    \item $\epsilon_r$ - \textbf{relative permittivity} or \textbf{dielectric constant}: Relative to the material, and its greater than 1 for all ordinary material.
\end{itemize}

\subsection{Fundamental quantities}
The 3 fundamental quantities are the charge density, $\rho$, the electric field, $\vb{E}$, and the potential, $V$, and they are related as follows

\begin{align*}
    \rho \to \vb{E} & :\quad \vb{E} = \ck \int \frac{\hat{\boldscriptr}}{\scriptr^2} \rho d\tau \\
    \vb{E} \to \rho & :\quad \nabla \cdot \vb{E} = \frac{\rho}{ \epsilon_0}; \, \nabla \times \vb{E} = \vb{0} \\
    \vb{E} \to V & :\quad V = -\int \vb{E} \cdot d\vb{l} \\
    V \to \vb{E} & :\quad \vb{E} = -\nabla V \\
    V \to \rho & :\quad \nabla^2 V = -\frac{\rho}{\epsilon_0}\\
    \rho \to V & :\quad V = \ck \int \frac{\rho}{\scriptr} d\tau
\end{align*}

\section{Magnetostatics}

Force is given by
$$
F_{mag} = Q[\vb{E} + (\vb{v} \times \vb{B})]
$$

Current is a vector, and is given by
$$
\vb{I} = \lambda \vb{v}
$$
for a line charge $\lambda$.  It is physically the charge per unit time.

The magnetic force can then be calculated in terms of the current
$$
\vb{F}_{mag} = \int I (d\vb{l} \times \vb{B})
$$

\begin{itemize}
    \item The hardest part here is generally finding the direction of the field/current using right hand rule
    \item Remember magnetic fields do no work, it's actually done by the battery
    \item Think about pushing a block up a ramp, the slope isn't doing the work, you are.
\end{itemize}

The surface current density is given by $K$ and represents current per unit width, and the volume current density is given by $J$ and represents the current per unit area:

$$
\vb{K} = \frac{d\vb{I}}{dl_\perp} = \sigma \vb{v} \qquad \vb{J} = \frac{d\vb{I}}{da_\perp} = \rho \vb{v}
$$

In sum, we have the following tools for translating equations from point, line, surface, and volume
$$
\sum_{i=1}^n (\,)q_i \vb{v}_i \sim \int_{line} (\,)\vb{I}dl \sim \int_{surface} (\,) \vb{K}da \sim \int_{volume} (\,) \vb{J}d\tau
$$

\subsection{Continuity equation derivation}

Re-arranging the definition of $\vb{J}$ and integrating we get
$$
I = \int_\mathcal{S} J da_\perp = \int_\mathcal{S} \vb{J} \cdot \vb{a}_\perp
$$
And Gauss's law states that the charge per unit time leaving the volume $\mathcal{V}$ is
$$
\oint_\mathcal{S} \vb{J} \cdot d\vb{a} = \int_\mathcal{V} (\nabla \cdot \vb{J}) d\tau
$$
Now, since charge is conserved, whatever flows out through the surface must come at the expense of what remains inside:
$$
\int_\mathcal{V} (\nabla \cdot \vb{J}) d\tau = -\frac{d}{dt} \int_\mathcal{V} \rho d\tau = - \int_\mathcal{V} \left( \frac{\partial \rho}{\partial t} \right) d\tau
$$
and since this applies for all volumes, we can set the inside of the integral equal and get the \textbf{continuity equation}
$$
\nabla \cdot \vb{J} = - \frac{\partial \rho}{\partial t}
$$

\subsection{Biot-Savart Law}
For a steady current, where
$$
\frac{\partial \vb{J}}{\partial t} = 0
$$
The magnetic field is given by
$$
\vb{B}(\vb{r}) = \frac{\mu_0}{4\pi} \int \frac{\vb{I} \times \vh{\boldscriptr}}{\scriptr^2} dl' = \frac{\mu_0}{4\pi} I \int \frac{d\vb{l'}\times \vh{\boldscriptr}}{\scriptr^2}
$$
where $\mu_0$ is the \textbf{permeability of free space}.

\subsection{Divergence and Curl}
From the above, for the case of a general volume current density, we can find that
$$
\nabla \cdot \vb{B} = 0
$$
and 
$$
\nabla \times \vb{B} = \mu_0 \vb{J}
$$
And applying stokes theorem, we get the equation for an \textbf{Amperian Loop}
$$
\oint \vb{B} \cdot d\vb{l} = \mu_0 I_{enc}
$$
which can give us the enclosed current.
\subsection{Magnetic Vector Potential}
Instead of a scalar potential like electrostatics, magnetostatics has a vector potential
$$
\vb{B} = \nabla \times \vb{A}
$$
With properties
$$
\nabla \cdot \vb{A} = 0 \qquad \nabla^2 \vb{A} = -\mu_0 \vb{J}
$$
and is explicitly calculated as 
$$
\vb{A}(\vb{r}) = \frac{\mu_0}{4\pi} \int \frac{\vb{J}(\vb{r'})}{\scriptr} d\tau
$$

\subsection{Boundary Conditions}
Griffiths Page 250

\subsection{Multipole Expansion}
Similar to electrostatics, we can expand the magnetic field as a power series of $1/r$. 

One important difference is that the dipole moment in electrostatics dominates ofnly if the electric monopole term is zero, but \textbf{the magnetic monopole term is always zero}.

Thus
$$
\vb{A}_{dip} (\vb{r}) = \frac{\mu_0}{4\pi} \frac{\vb{m}\times\vh{r}}{r^2}
$$
where $\vb{m}m$ is the magnetic dipole moment
$$
\vb{m} = I \int d\vb{a} = I\vb{a}
$$

\subsection{Fundamental Quantities}
The 3 fundamental quantities are the current density, $\vb{J}$, the magnetic field $\vb{B}$, and the magnetic potential, $\vb{A}$, and they are related as follows.

\begin{align*}
    \vb{J} \to \vb{B} & :\quad \vb{B} = \frac{\mu_0}{4\pi} \int \frac{\vb{J} \times \hat{\boldscriptr}}{\scriptr^2} d\tau \\
    \vb{B} \to \vb{J} & :\quad \nabla \times \vb{B} = \mu_0 \vb{J}; \, \nabla \cdot \vb{B} = \vb{0} \\
    \vb{B} \to \vb{A} & : \quad \text{This does not exist} \\
    \vb{A} \to \vb{B} & :\quad \vb{B} = \nabla \times \vb{A}; \, \nabla \cdot \vb{A} = 0 \\
    \vb{A} \to \vb{J} & :\quad \nabla^2 \vb{A} = -\mu_0 \vb{J}\\
    \vb{J} \to \vb{A} & :\quad \vb{A} = \frac{\mu_0}{4\pi} \int \frac{\vb{J}}{\scriptr} d\tau
\end{align*}

\section{Magnetic Fields in Matter}









\end{document}
